\section{Введение}

В последние десятилетия задачи анализа и обработки качественных и неопределённых данных всё чаще решаются методами {\it мягких вычислений}, в основе которых лежат нечеткие множества и нечеткая логика. Их популярность обусловлена способностью адекватно моделировать лингвистическую и стохастическую неопределённость, характерную для реальных экспертных оценок и измерений. Идея нечетких множеств, предложенная Л. Заде, заключается в описании принадлежности элементов к понятиям не «0 или 1», а через значения в интервале $[0,1]$, что позволяет более гибко отражать размытые границы и переходные состояния реальных явлений.

Ключевым этапом построения нечетких систем является нечеткий вывод — процедура, которая на основании базы нечётких правил и текущих (в том числе нечетких) входных данных формирует выходное заключение. Классический метод Л. Заде с композиционным правилом вывода (Compositional Rule of Inference) имеет строгую теоретическую базу, но при увеличении числа входных переменных сталкивается с экспоненциальным ростом вычислений. Более практичны методы Мамдани или Такаги–Сугено, обладающие полиномиальной сложностью, но порой упрощающие исходную модель (например, синглтонная фаззификация), что может приводить к потере информации о «размытой» природе входов.

Обычные системы применяют фаззификацию синглтонными функциями принадлежности, сводя каждую входную оценку к одному числу. Однако во многих реальных задачах экспертные данные и сенсорные измерения содержат шум и лингвистическую неопределённость, для которых более естественна {\it несинглтонная фаззификация}. При ней каждое наблюдение описывается распределённым по отрезку нечётким множеством, что сохраняет информацию о возможном разбросе значений и повышает устойчивость модели.

\subsection{Актуальность задачи}

Интеллектуальные системы во многих областях — от управления технологическими процессами до анализа социально-экономических показателей — требуют одновременно:
\begin{itemize}
  \item учёта лингвистической неопределённости и «нечёткости» экспертных оценок;
  \item высокой интерпретируемости и прозрачности принятия решений;
  \item возможности эффективной работы при малых обучающих выборках;
  \item полиномиальной вычислительной сложности при большом числе входных параметров.
\end{itemize}

Нечёткие системы удовлетворяют всем этим требованиям, тогда как современные нейронные сети зачастую нуждаются в больших объёмах данных, обладают «чёрным ящиком» внутри и требуют значительных вычислительных ресурсов для обучения и вывода. Преимущества нечеткой логики по сравнению с нейросетевыми подходами заключаются в следующем:
\begin{itemize}
  \item \textbf{Интерпретируемость.} Правила вида «Если $A$ — высокое, а $B$ — среднее, то $C$ — низкое» легко читаются и проверяются человеком, в то время как внутренние веса нейросети не дают прозрачного объяснения решения.
  \item \textbf{Работа с малыми данными.} Нечёткие системы могут строиться на экспертных знаниях и требуют лишь описания функций принадлежности и набора правил, тогда как нейросети часто нуждаются в тысячах примеров для адекватного обучения.
  \item \textbf{Устойчивость к неопределённости.} При несинглтонной фаззификации сохраняется информация о разбросе входов, что повышает стабильность вывода при шумных измерениях.
  \item \textbf{Низкие вычислительные затраты на этапе вывода.} При грамотно подобранной реализации и полиномиальном алгоритме нечёткий вывод обходится значительно дешевле, чем многослойные сети, особенно в системах с жёсткими требованиями к времени реакции.
\end{itemize}

\subsection{Цель и задачи исследования}

\paragraph{Цель исследования.} 

Повышение эффективности анализа качественных и неопределённых данных за счёт разработки и практической апробации нечетких систем с несинглтонной фаззификацией и полиномиально сложным алгоритмом вывода, сочетающих интерпретируемость и скорость работы.

\paragraph{Для достижения цели ставятся следующие задачи:}
\begin{enumerate}
  \item Провести обзор и критический анализ существующих методов нечеткого вывода и фаззификации, в том числе классических (Заде), Мамдани, Такаги–Сугено, а также подходов с несинглтонной фаззификацией;
  \item Разработать математический аппарат вывода на основе понятия нечеткого значения истинности, обеспечивающий полиномиальную сложность при многих нечетких входах;
  \item Предложить методы нечёткой классификации и регрессии объектов на основе распределённых входных множеств и экспертных правил;
  \item Создать программное обеспечение с удобным интерфейсом и возможностью аппаратного ускорения (GPU/CUDA) для эффективного применения предложенных методов;
  \item Провести серийные эксперименты на задачах управления техническими процессами и социально-экономического моделирования, сравнить результаты с альтернативными методами, включая нейросетевые.
\end{enumerate}

\subsection{Научная новизна и практическая значимость}

\paragraph{Научная новизна.}

\begin{itemize}
  \item Обосновано применение несинглтонной фаззификации в сочетании с обобщённым правилом Заде, что обеспечивает баланс между полнотой модели и вычислительной эффективностью;
  \item Разработан полиномиальный алгоритм нечеткого вывода на основе нечеткого значения истинности, сохраняющий interpretability и теоретическую строгость;
  \item Предложены новые методы нечёткой классификации и регрессии, демонстрирующие устойчивость к шуму и малым выборкам;
  \item Созданы параллельные реализации алгоритмов вывода и фаззификации, оптимизированные для GPU.
\end{itemize}

\paragraph{Практическая значимость.} 

Результаты работы позволят:
\begin{itemize}
  \item Быстро и прозрачно строить модели для управления и принятия решений в условиях неопределённости;
  \item Экономить ресурсы на этапе вывода благодаря полиномиальному алгоритму и аппаратному ускорению;
  \item Интегрировать нечёткие компоненты в существующие информационные системы без необходимости глубоких знаний машинного обучения.
\end{itemize}

\subsection{Структура работы}

Работа состоит из четырёх глав и заключения:
\begin{itemize}
  \item \textbf{Глава 1.} Обзор методов нечеткого вывода и фаззификации, введение понятия нечеткого значения истинности.
  \item \textbf{Глава 2.} Разработка полиномиального алгоритма вывода и методов классификации на основе несинглтонной фаззификации.
  \item \textbf{Глава 3.} Программная реализация предложенных методов с аппаратным ускорением.
  \item \textbf{Глава 4.} Экспериментальная оценка и сравнение с нейросетевыми и классическими подходами.
  \item \textbf{Заключение.} Основные выводы и рекомендации по дальнейшим исследованиям.
\end{itemize}
