
% \let\oldsection\section
% \renewcommand{\section}[1]{\oldsection{#1}\ESKDsignature{\normalfont\small #1}\eskdrerun}
\ESKDthisStyle{formII}
% \ESKDsignature{Введение}
\section*{Введение}
\addcontentsline{toc}{section}{Введение}
% \eskdrerun{}
\ESKDcolumnII{Введение}


В последние десятилетия задачи анализа и обработки качественных и неопределённых данных всё чаще решаются методами {\it мягких вычислений}, в основе которых лежат нечеткие множества и нечеткая логика. Их популярность обусловлена способностью адекватно моделировать лингвистическую и стохастическую неопределённость, характерную для реальных экспертных оценок и измерений. Идея нечетких множеств, предложенная Л. Заде, заключается в описании принадлежности элементов к понятиям не «0 или 1», а через значения в интервале $[0,1]$, что позволяет более гибко отражать размытые границы и переходные состояния реальных явлений.

Ключевым этапом построения нечетких систем является нечеткий вывод — процедура, которая на основании базы нечётких правил и текущих (в том числе нечетких) входных данных формирует выходное заключение. Классический метод Л. Заде с композиционным правилом вывода (Compositional Rule of Inference) имеет строгую теоретическую базу, но при увеличении числа входных переменных сталкивается с экспоненциальным ростом вычислений. Более практичны методы Мамдани или Такаги–Сугено, обладающие полиномиальной сложностью, но порой упрощающие исходную модель (например, синглтонная фаззификация), что может приводить к потере информации о «размытой» природе входов.

Обычные системы применяют фаззификацию синглтонными функциями принадлежности, сводя каждую входную оценку к одному числу. Однако во многих реальных задачах экспертные данные и сенсорные измерения содержат шум и лингвистическую неопределённость, для которых более естественна {\it несинглтонная фаззификация}. При ней каждое наблюдение описывается распределённым по отрезку нечётким множеством, что сохраняет информацию о возможном разбросе значений и повышает устойчивость модели.

\paragraph{Цель и задачи}

\textbf{Цель:} Повышение эффективности анализа качественных и неопределённых данных за счёт разработки и практической апробации нечетких систем с несинглтонной фаззификацией и полиномиально сложным алгоритмом вывода, сочетающих интерпретируемость и скорость работы.

\paragraph{Для достижения цели ставятся следующие задачи:}
\begin{enumerate}
  \item Провести обзор и критический анализ существующих методов нечеткого вывода и фаззификации, в том числе классических (Заде), Мамдани, Такаги–Сугено, а также подходов с несинглтонной фаззификацией;
  \item Разработать математический аппарат вывода на основе понятия нечеткого значения истинности, обеспечивающий полиномиальную сложность при многих нечетких входах;
  \item Предложить методы нечёткой классификации и регрессии объектов на основе распределённых входных множеств и экспертных правил;
  \item Создать программное обеспечение с удобным интерфейсом и возможностью аппаратного ускорения (GPU/CUDA) для эффективного применения предложенных методов;
  \item Провести серийные эксперименты на задачах управления техническими процессами и социально-экономического моделирования, сравнить результаты с альтернативными методами, включая нейросетевые.
\end{enumerate}

\paragraph{Научная новизна}

\begin{itemize}
  \item Обосновано применение несинглтонной фаззификации в сочетании с обобщённым правилом Заде, что обеспечивает баланс между полнотой модели и вычислительной эффективностью;
  \item Разработан полиномиальный алгоритм нечеткого вывода на основе нечеткого значения истинности, сохраняющий interpretability и теоретическую строгость;
  \item Предложены новые методы нечёткой классификации и регрессии, демонстрирующие устойчивость к шуму и малым выборкам;
  \item Созданы параллельные реализации алгоритмов вывода и фаззификации, оптимизированные для GPU.
\end{itemize}

\paragraph{Практическая значимость} 

Результаты работы позволят:
\begin{itemize}
  \item Быстро и прозрачно строить модели для управления и принятия решений в условиях неопределённости;
  \item Экономить ресурсы на этапе вывода благодаря полиномиальному алгоритму и аппаратному ускорению;
  \item Интегрировать нечёткие компоненты в существующие информационные системы без необходимости глубоких знаний машинного обучения.
\end{itemize}

