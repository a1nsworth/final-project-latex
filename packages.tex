\usepackage[T2A]{fontenc}
\usepackage[utf8]{inputenc}
\usepackage[russian]{babel}

\usepackage{listings}
\usepackage{xcolor}

\definecolor{codebg}{HTML}{F8F8F8}
\definecolor{keyword}{RGB}{0,0,180}
\definecolor{string}{RGB}{163,21,21}
\definecolor{comment}{RGB}{0,128,0}
\definecolor{builtin}{RGB}{43,145,175}
\definecolor{type}{RGB}{127,0,85}
\definecolor{number}{RGB}{128,0,128}

% --- стиль listings ---
\lstdefinestyle{bright}{
  backgroundcolor=\color{codebg},
  basicstyle=\ttfamily\small,
  keywordstyle=\color{keyword}\bfseries,
  commentstyle=\color{comment}\itshape,
  stringstyle=\color{string},
  identifierstyle=\color{black},
  emph={int,float,double,char,void,bool},            
  emphstyle=\color{type}\bfseries,                    
  morekeywords=[2]{torch,Tensor,torch.nn,torch.optim,cudaMalloc,cudaMemcpy}, 
  keywordstyle=[2]\color{builtin},                    
  numberstyle=\tiny\color{number},
  stepnumber=1,
  numbersep=6pt,
  frame=single,
  rulecolor=\color{gray!50},
  showspaces=false,
  showstringspaces=false,
  breaklines=true,
  breakatwhitespace=true,
  tabsize=2,
  columns=fullflexible,
  captionpos=b,
}
\lstset{style=bright}

\usepackage{chngcntr}
\counterwithin{figure}{section}
\counterwithin{table}{section}

\usepackage{setspace}
\onehalfspacing

\usepackage{graphicx}   
\usepackage{longtable} 
\usepackage{amsmath}  
\usepackage{titlesec}

\usepackage{tabularx} 

\usepackage{amsmath, amssymb, amsfonts, amsthm} % для \Diamond, \tilde и т.д.
\usepackage{pgfplots}                   % если оставляете \begin{tikzpicture}
\usepackage{tikz}
\usetikzlibrary{positioning, shapes, arrows.meta, fit, matrix}

\usepackage{booktabs}  
\pgfplotsset{compat=1.17}
\usetikzlibrary{calc} 

\usepackage{hyperref}

\theoremstyle{definition}
\newtheorem{definition}{Определение}[section]

\newtheorem{theorem}{Теорема}[section]
\newtheorem{lemma}[theorem]{Лемма}

\theoremstyle{remark}
\newtheorem*{remark}{Замечание}
