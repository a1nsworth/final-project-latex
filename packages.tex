%--- Кодировка и локализация ------------------------------------
% \usepackage[T2A]{fontenc}
\usepackage[utf8]{inputenc}
\usepackage[russian]{babel}

\usepackage{lastpage}
\usepackage{tempora} 
\usepackage{newtxmath}

% \usepackage{pscyr}
% \renewcommand{\rmdefault}{ftm}

%--- Математика --------------------------------------------------
\usepackage{amsmath, amsfonts, amsthm}
% \usepackage{mathtools}               % расширения для amsmath

\DeclareMathOperator{\CP}{CP}

%--- Графика и рисунки ------------------------------------------
\usepackage{graphicx}                % вставка растровых и векторных изображений
\usepackage{tikz}                    % рисование диаграмм, схем и т.п.
\usetikzlibrary{
  positioning,
  shapes,
  arrows.meta,
  fit,
  matrix,
  calc
}
\usepackage{pgfplots}                % построение графиков
\pgfplotsset{compat=1.17}

%--- Таблицы ----------------------------------------------------
\usepackage{longtable}               % большие таблицы
\usepackage{tabularx}                % табличный текст с автоподбором ширины
\usepackage{booktabs}                % красивые горизонтальные линии

%--- Счётчики ---------------------------------------------------
\usepackage{chngcntr}
\counterwithin{figure}{section}       % нумерация рисунков по секциям
\counterwithin{table}{section}        % нумерация таблиц по секциям

%--- Оформление заголовков --------------------------------------
\usepackage{titlesec}

%--- Междустрочный интервал -------------------------------------
\usepackage{setspace}
\onehalfspacing                      % полуторный интервал

%--- Работа с исходным кодом -----------------------------------
\usepackage{listings}
\usepackage{xcolor}
% Цветовые определения для listings
\definecolor{codebg}{HTML}{F8F8F8}
\definecolor{keyword}{RGB}{0,0,180}
\definecolor{string}{RGB}{163,21,21}
\definecolor{comment}{RGB}{0,128,0}
\definecolor{builtin}{RGB}{43,145,175}
\definecolor{type}{RGB}{127,0,85}
\definecolor{number}{RGB}{128,0,128}

% Стиль для listings
\lstdefinestyle{bright}{
  backgroundcolor=\color{codebg},
  basicstyle=\ttfamily\small,
  keywordstyle=\color{keyword}\bfseries,
  commentstyle=\color{comment}\itshape,
  stringstyle=\color{string},
  identifierstyle=\color{black},
  emph={int,float,double,char,void,bool},
  emphstyle=\color{type}\bfseries,
  morekeywords=[2]{torch,Tensor,torch.nn,torch.optim,cudaMalloc,cudaMemcpy},
  keywordstyle=[2]\color{builtin},
  numberstyle=\tiny\color{number},
  stepnumber=1,
  numbersep=6pt,
  frame=single,
  rulecolor=\color{gray!50},
  showspaces=false,
  showstringspaces=false,
  breaklines=true,
  breakatwhitespace=true,
  tabsize=2,
  columns=fullflexible,
  captionpos=b
}
\lstset{style=bright}

%--- Гиперссылки -----------------------------------------------
\usepackage{hyperref}

\theoremstyle{definition}
\newtheorem{definition}{Определение}[section]

\theoremstyle{plain}
\newtheorem{theorem}{Теорема}[section]
\newtheorem{lemma}[theorem]{Лемма}

\theoremstyle{remark}
\newtheorem*{remark}{Замечание}

\usepackage{natbib}



\makeatletter
\let\origparagraph\paragraph
\renewcommand{\paragraph}[1]{%
  \origparagraph{#1}\mbox{}\\ 

  %
}
\makeatother
