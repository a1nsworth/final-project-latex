% --- Кодировка и локализация ------------------------------------
\usepackage[T2A]{fontenc}
\usepackage[utf8]{inputenc}
\usepackage[russian]{babel}

% 1) Загружаем minted как float-пакет
\usepackage[float]{minted}


% 4) Меняем формат номера: <номер секции>.<номер листинга>
\renewcommand{\thelisting}{\thesection.\arabic{listing}}

\usepackage{xcolor}
\definecolor{codebg}{RGB}{245,247,250}

\usepackage{minted}
\setminted{%
  style=xcode,            % мягкая, контрастная подсветка
  bgcolor=codebg,         % фон блока
  frame=single,           % тонкая рамка вокруг
  framerule=0.5pt,        % толщина рамки
  framesep=2mm,           % отступ текста от рамки
  rulecolor=\color{gray!60}, % цвет рамки
  linenos,                % нумерация строк
  numbersep=5pt,          % расстояние номера от кода
  xleftmargin=2em,        % дополнительный отступ слева
  autogobble,             % обрезаем общий отступ кода
  fontsize=\footnotesize, % компактный шрифт
  baselinestretch=1,      % одинарный межстрочный
  breaklines,             % автоматически переносить длинные строки
  breakindent=1em,        % отступ для перенесённых строк
  tabsize=2               % табуляция = 2 пробела
}
% -------------------------------------------------

\usepackage{lastpage}
\usepackage{newtxmath}

% \usepackage{pscyr}
% \renewcommand{\rmdefault}{ftm}

%--- Математика --------------------------------------------------
\usepackage{amsmath, amsfonts, amsthm}
% \usepackage{mathtools}               % расширения для amsmath

\DeclareMathOperator{\CP}{CP}

%--- Графика и рисунки ------------------------------------------
\usepackage{caption}
\usepackage{float}
\usepackage{graphicx}                % вставка растровых и векторных изображений
\usepackage{tikz}                    % рисование диаграмм, схем и т.п.
\usetikzlibrary{
  positioning,
  shapes,
  arrows.meta,
  fit,
  matrix,
  calc
}
\usepackage{pgfplots}                % построение графиков
\pgfplotsset{compat=1.17}

\usetikzlibrary{patterns,intersections,fillbetween}
\usepgfplotslibrary{fillbetween}

%--- Таблицы ----------------------------------------------------
\usepackage{longtable}               % большие таблицы
\usepackage{tabularx}                % табличный текст с автоподбором ширины
\usepackage{booktabs}                % красивые горизонтальные линии

%--- Счётчики ---------------------------------------------------
\usepackage{chngcntr}
\counterwithin{figure}{section}       % нумерация рисунков по секциям
\counterwithin{table}{section}        % нумерация таблиц по секциям

%--- Оформление заголовков --------------------------------------
\usepackage{titlesec}
\titleclass{\subsubsubsection}{straight}[\subsubsection]
\newcounter{subsubsubsection}[subsubsection]
\renewcommand\thesubsubsubsection%
  {\thesubsubsection.\arabic{subsubsubsection}}

% задаём формат: жирный нормальный размер, отступ 1em после номера
\titleformat{\subsubsubsection}
  {\normalfont\normalsize\bfseries}
  {\thesubsubsubsection}{1em}{}
%--- Междустрочный интервал -------------------------------------
\usepackage{setspace}
\onehalfspacing                      % полуторный интервал

%--- Работа с исходным кодом -----------------------------------
\usepackage{listings}
\usepackage{xcolor}
% Цветовые определения для listings
\definecolor{codebg}{HTML}{F8F8F8}
\definecolor{keyword}{RGB}{0,0,180}
\definecolor{string}{RGB}{163,21,21}
\definecolor{comment}{RGB}{0,128,0}
\definecolor{builtin}{RGB}{43,145,175}
\definecolor{type}{RGB}{127,0,85}
\definecolor{number}{RGB}{128,0,128}

% Стиль для listings
\lstdefinestyle{bright}{
  inputencoding=utf8,
  extendedchars=true,
  backgroundcolor=\color{codebg},
  basicstyle=\ttfamily\small,
  keywordstyle=\color{keyword}\bfseries,
  commentstyle=\color{comment}\itshape,
  stringstyle=\color{string},
  identifierstyle=\color{black},
  emph={int,float,double,char,void,bool},
  emphstyle=\color{type}\bfseries,
  morekeywords=[2]{torch,Tensor,torch.nn,torch.optim,cudaMalloc,cudaMemcpy},
  keywordstyle=[2]\color{builtin},
  numberstyle=\tiny\color{number},
  stepnumber=1,
  numbersep=6pt,
  frame=single,
  rulecolor=\color{gray!50},
  showspaces=false,
  showstringspaces=false,
  breaklines=true,
  breakatwhitespace=true,
  tabsize=2,
  columns=fullflexible,
  captionpos=b
}
\lstset{style=bright}

%--- Гиперссылки -----------------------------------------------
\usepackage{hyperref}

\theoremstyle{definition}
\newtheorem{definition}{Определение}[section]

\theoremstyle{plain}
\newtheorem{theorem}{Теорема}[section]
\newtheorem{lemma}[theorem]{Лемма}

\theoremstyle{remark}
\newtheorem*{remark}{Замечание}

\newtheorem{example}{Пример}[section]

\usepackage{natbib}

%--- КОД ---------------------------------------------------------
\makeatletter
\let\origparagraph\paragraph
\renewcommand{\paragraph}[1]{%
  \origparagraph{#1}\mbox{} 
  %
}
\makeatother


\makeatletter
\let\origsubparagraph\subparagraph
\renewcommand{\subparagraph}[1]{%
  \origsubparagraph{#1}\mbox{} 
  %
}
\makeatother

