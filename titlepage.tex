\ESKDthisStyle{empty}

% ---------- титульный лист ----------
\begin{center}
	\textbf{МИНОБРНАУКИ РОССИИ}\\[0.5em]
	\vspace{0.2em}
	{\scriptsize ФЕДЕРАЛЬНОЕ ГОСУДАРСТВЕННОЕ БЮДЖЕТНОЕ ОБРАЗОВАТЕЛЬНОЕ УЧРЕЖДЕНИЕ\\[-0.5em]
		ВЫСШЕГО ОБРАЗОВАНИЯ}\\[0.5em]
	{\small \textbf{«БЕЛГОРОДСКИЙ ГОСУДАРСТВЕННЫЙ ТЕХНОЛОГИЧЕСКИЙ\\
		УНИВЕРСИТЕТ им.\ В.\,Г.\ ШУХОВА»\\[-0.3em]
		(БГТУ им.\ В.\,Г.\ Шухова)}}
\end{center}

\vspace{1em}

% ---------- реквизиты ----------
\begin{flushleft}
	{\footnotesize                           % ← уже была группа
		\Signature{Институт:}{}{{\scriptsize \textit{энергетики, информационных технологий и управляющих систем}}}{}\\
		\Signature{Кафедра:}{}{\scriptsize \textit{обеспечения вычислительной техники и автоматизированных систем}}{}\\
		\Signature{Направление подготовки:}{(шифр, наименование)}{\scriptsize \textit{09.03.01 Информатика и вычислительная техника}}{}\\
		\Signature{Направленность образовательной программы:}{(наименование)}{\scriptsize \textit{Разработка программно-информационных систем}}{}%
	}
\end{flushleft}

\vspace{1em}

\begin{center}
	\textbf{ВЫПУСКНАЯ КВАЛИФИКАЦИОННАЯ РАБОТА}\\[0.1em]
	на тему:\\[0.5em]
	\textbf{Разработка программной системы для нечеткого моделирования с non\-singleton входными данными}
\end{center}

\vspace{1em}

\begin{flushright}
	\small
	\begin{minipage}{7cm}
		Студент: ст. группы ВТ-212 \\ Уахби Д.\ А\\
		Зав. кафедрой: канд.\ техн.\ наук, доц.\ \\ Поляков В.\ М\\
		Руководитель: ст.\ преподаватель \\ Панченко М.\ В
	\end{minipage}
\end{flushright}

\begin{center}
	\begin{minipage}{11cm}
		К защите допустить\\[1em]
		Зав.\ кафедрой \underline{\hspace{4cm}} / Поляков В.\ М\\[1em]
		«\underline{\hspace{1cm}}» \underline{\hspace{4cm}} 2025 г.
	\end{minipage}
\end{center}

\vfill
\begin{center}
	Белгород 2025 г.
\end{center}

\newpage
\ESKDthisStyle{empty}

% ---------- лист «Задание» ----------
\begin{center}
	\textbf{МИНОБРНАУКИ РОССИИ}\\[0.5em]
	\vspace{0.2em}
	{\scriptsize ФЕДЕРАЛЬНОЕ ГОСУДАРСТВЕННОЕ БЮДЖЕТНОЕ ОБРАЗОВАТЕЛЬНОЕ УЧРЕЖДЕНИЕ\\[-0.5em]
		ВЫСШЕГО ОБРАЗОВАНИЯ}\\[0.5em]
	{\small \textbf{«БЕЛГОРОДСКИЙ ГОСУДАРСТВЕННЫЙ ТЕХНОЛОГИЧЕСКИЙ\\
		УНИВЕРСИТЕТ им.\ В.\,Г.\ ШУХОВА»\\[-0.3em]
		(БГТУ им.\ В.\,Г.\ Шухова)}}
\end{center}

\vspace{1em}

\begin{flushleft}
	{\footnotesize
		\Signature{Институт:}{}{{\scriptsize \textit{энергетики, информационных технологий и управляющих систем}}}{}\\
		\Signature{Кафедра:}{}{\scriptsize \textit{обеспечения вычислительной техники и автоматизированных систем}}{}\\
		\Signature{Направление подготовки:}{(шифр, наименование)}{\scriptsize \textit{09.03.01 Информатика и вычислительная техника}}{}\\
		\Signature{Направленность образовательной программы:}{(наименование)}{\scriptsize \textit{Разработка программно-информационных систем}}{}
	}
\end{flushleft}

\begin{flushright}
	\begin{minipage}{8.5cm}
		\scriptsize
		Утверждаю:\\
		Зав.\ кафедрой \underline{\hspace{4cm}} / Поляков В.\ М\\[0.5em]
		«\underline{\hspace{1cm}}» \underline{\hspace{4cm}} 2025 г.
	\end{minipage}
\end{flushright}

\vspace{1cm}

\begin{center}
	\textbf{ЗАДАНИЕ}\\[0.1em]
	на выпускную квалификационную работу студента\\
	\Signature{}{(Фамилия, Имя, Отчество)}{\hspace{5cm}Уахби Даниэля Абдулаховича\hspace{5cm}}{}
\end{center}

% ---------- список требований ----------
\begin{footnotesize}% ← теперь шрифт ограничен только этим блоком
	\begin{enumerate}[label=\arabic*.]
		\item Вид выпускной квалификационной работы (ВКР): \textit{бакалаврская работа.}
		\item Тема ВКР: \textit{Разработка программной системы для нечеткого моделирования с non-singleton входными данными.}
		\item Срок сдачи студентом законченной ВКР:
		\item Исходные данные: нечеткая логика, нейро-нечеткая система, нечеткое значение истинности, нечеткая степень истинности.
		\item Содержание ВКР (перечень подлежащих разработке разделов): \textit{список сокращений, введение, описание предметной области, анализ и выбор метода решения задач, проектирование программного обеспечения, программная реализация, список литературы, приложение «А».}
		\item Перечень графического материала: \textit{титульный лист, цель и задачи, обоснование и актуальность, сравнение методов вывода в нечетких системах, метод нечеткого вывода на основе нечеткого значения истинности, сравнение алгоритмов настройки параметров нечетких систем, сетевая структура подсистемы нечеткого вывода, сравнение результатов нечеткого вывода с аналогами non-singleton-систем, результат работы, заключение по проделанной работе.}
	\end{enumerate}
\end{footnotesize}

\newpage
\ESKDthisStyle{empty}

% ---------- консультанты ----------
Консультанты по работе с указанием относящихся к ним разделов

\begin{center}
	\setlength\tabcolsep{4pt}
	\begin{tabular}{|p{4cm}|p{3cm}|p{4cm}|p{4cm}|}
		\hline
		\centering Раздел      &
		\centering Консультант &
		\makecell[c]{Задание выдал  \\\footnotesize(подпись, дата)} &
		\makecell[c]{Задание принял \\\footnotesize(подпись, дата)} \\ \hline
	\end{tabular}
\end{center}

\vspace{3cm}
\begin{center}
	\footnotesize
	Дата выдачи задания «\underline{\hspace{1cm}}» \underline{\hspace{2cm}} 2025 г.\\[4em]
	\begin{minipage}{15cm}
		\scriptsize
		\Signature{Руководитель}{(подпись)}{\hspace{3cm}}{/ Панченко М.\ В}\hspace{2em}
		\Signature{Студент}{(подпись)}{\hspace{3cm}}{/ Уахби Д.\ А}
	\end{minipage}
\end{center}

\vspace{3cm}

% ---------- календарный план ----------
\begin{center}
	\textbf{Календарный план}\\[1cm]
	\setlength\tabcolsep{4pt}
	\begin{tabular}{|p{1cm}|p{6cm}|p{4cm}|p{3cm}|}
		\hline
		\makecell[c]{№                                                                             \\п/п} &
		\centering Наименование этапов работы    &
		\centering Срок выполнения этапов работы &
		\centering Примечание                    & \hline
		1                                        & Анализ предметной области                     &
		\hspace{2cm}                             & Выполнено                                       \\ \hline
		2                                        & Проектирование программного обеспечения       &
		\hspace{2cm}                             & Выполнено                                       \\ \hline
		3                                        & Программная реализация приложения             &
		\hspace{2cm}                             & Выполнено                                       \\ \hline
		4                                        & Оформление пояснительной записки, презентация &
		\hspace{2cm}                             & Выполнено                                       \\ \hline
	\end{tabular}

	\vspace{4cm}
	\begin{minipage}{15cm}
		\scriptsize
		\Signature{Руководитель}{(подпись)}{\hspace{3cm}}{/ Панченко М.\ В}\hspace{2em}
		\Signature{Студент}{(подпись)}{\hspace{3cm}}{/ Уахби Д.\ А}
	\end{minipage}
\end{center}
